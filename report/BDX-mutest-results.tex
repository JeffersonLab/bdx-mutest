\section{Results}\label{sec:results}

Muons and beam-related background  were produced by the 11 GeV electron beam interaction with the beam-dump and propagated in the region of interest as described in  Sec.~\ref{sec:sampling}.

\begin{figure}[h!] 
\center
\includegraphics[width=4.7cm]{figs/comparisonMuonsPipe1_1D.pdf}
\includegraphics[width=4.7cm]{figs/comparisonMuonsPipe2_1D.pdf}
\includegraphics[width=5.5cm]{figs/comparisonMuonsPipe3_1D.pdf}
\caption{Muons energy spectra at the three locations of interest (A, B and C). Beam/dump interaction using  FLUKA (black), GEMC (red) and the high statistic custom $\mu$ event generator with GEMC propagation (green).}
\label{fig:mu-comp}
\end{figure}


\subsection{Muon detection}
Fig.~\ref{fig:mu-comp} shows the muon flux crossing the BDX-Hodo  as obtained by GEMC and FLUKA  in the  three locations of interest ({\bf A}, {\bf B} and {\bf C}).
The flux has been sampled assuming the BDX-Hodo centred  to the beam height  Results are reported for muons generated at high statistic by the custom $\mu$ event generator and propagated using GEMC (green points in the figure).
The number of event generated at the dump correspond to  (1.2 $\pm$0.1) 10$^{12}$ EOT or one second of 0.2 $\mu$A current.
Rates in the crystals, in the scintillators and in requiring a 5-fold coincidence of the two front/back layers of plastic with the crystals are reported in Tab.~\ref{tab: rate} assuming a beam current of 10$\mu$A. Results show a drop in rate by about one order of magnitude when moving from one location to the next. 
Fluxes in position {\bf C} (or/and {\bf B}) are big enough to be easily measurable (against cosmic muons and beam-dump neutron background) and handle by the front end electronics (no pile-up effects expected). These two locations are the closest to the paved road and easily accessible by the drilling machine and related equipment. Same results are valid if the beam current drops/increases by one order of magnitude (1/100 $\mu$A) making the test feasible in parallel to any 11 GeV operation of Hall-A.

\begin{table}[htp]
\caption{Beam-dump muon rates expected in BDX-Hodo for I$_{beam}$=10$\mu A$.}
\begin{center}
\begin{tabular}{|c|c|c|c|c|}
\hline
Location & Rate$_{Crystal}$  (kHz)&  Rate$_{Front-Back \;Scint} $(kHz) & Rate$_{Coin}$ (kHz) & Rate$_{XY\, ch} $(kHz)\\
\hline\hline
{\bf A}  &   &    &  & \\
 \hline
{\bf B} & 20 &  15/30 & 3.7 & 0.7\\
 \hline
{\bf C} & 2.8 &  1/2.5 & 0.5 & 0.1 \\
\hline\hline
\end{tabular}
\end{center}
\label{tab:rate}
\end{table}%



\subsection{Muon flux above the ground}
For sake of completeness the muon flux has also been evaluated in the closest locations accessible above the  beam-dump vault by using FLUKA.  This set-up assumes to locate the detector above-the-ground with no drill required  and simplifying the logistic of the tests.
Due to the CPU-time necessary to track muons at such large angle (with respect to the beam axis) we used a two steps procedure. Firstly, the 11 GeV electron beam was let interact with the beam dump and muons produced on the roof of the vault has been sampled in three different position. Figure~\ref{fig:bd-top}-left shows the four positions on the roof of the beam-dump vault where the flux has been sampled.
A high statistic sample of muons have then been generated according to the previous distributions and propagated to the outside.  Applying  a conservative hypothesis (the muons are propagated perpendicular to the beam axis  crossing the minimal amount of concrete and dirt) muons with energy higher than 4.5 GeV (the minimum to not been ranged out)  were propagated and sampled in the three perpendicular locations outdoor. Figure~\ref{fig:bd-top}-right shows the four locations (A$_{Ext}$, B$_{Ext}$, C$_{Ext}$,  and D$_{Ext}$) on top of the hill. When integrated over the surface of the BDX-Hodo detector ($\sim 100$ cm$^2$) and considering as a reference a beam current of 10 $\mu$A, no sizeable muon flux would be detected (Rate$_{Max}<$3 Hz)
\begin{figure}[h!] 
\center
\includegraphics[width=8.0cm]{figs/DumpTunnelTopBigger1.pdf}
\includegraphics[width=5.7cm]{figs/DumpTunnelTop1.pdf}
\caption {Left: In red are shown the points on the roof  where muon flux has been sampled. Right:  A$_{Ext}$, B$_{Ext}$, C$_{Ext}$,  and D$_{Ext}$ are the four location outdoor where the muon flux has been evaluated.}
\label{fig:bd-top}
\end{figure}
Table~\ref{tab:outside} report the rates as obtained by the simulations.

\begin{table}[htp]
\caption{Cosmic rate expected in different components of BDX-Hodo}
\begin{center}
\begin{tabular}{|c|c|}
\hline
Location &Rate$_{Crystal} $ (Hz / (cm$^2$ $\mu$A) \\
\hline\hline
A$_{Ext} $ & 2.2 $10^{-3}$\\
 \hline
B$_{Ext} $ & 4.7 $10^{-4}$\\
 \hline
C$_{Ext} $ & 1.9 $10^{-3}$\\
 \hline
D$_{Ext}$ & negligible\\

\hline\hline
\end{tabular}
\end{center}
\label{tab:outside}
\end{table}%




\subsubsection{Beam-related background}
Beside muons, other particles are produced in the 11 GeV electron beam interaction with the dump.
The majority (electrons, gamma, nuclei and fragments) are ranged out well before to reach the region of interest but some (low energy neutrons mainly) may propagate thought the concrete and the dirt reaching the BDX-Hodo detector.
Fig.~\ref{fig:nu-comp} shows the neutron flux  as obtained by  FLUKA starting from the electron/beam-dump interaction  in the  three locations of interest. 
\begin{figure}[h!] 
\center
\includegraphics[width=4.7cm]{figs/NeutronsPipe1_1D.pdf}
\includegraphics[width=4.7cm]{figs/NeutronsPipe2_1D.pdf}
\includegraphics[width=4.7cm]{figs/NeutronsPipe3_1D.pdf}
\caption {Neutron energy spectra at the three locations of interest. Spectra are obtained from electron beam interaction with the beam-dump using FLUKA.}
\label{fig:nu-comp}
\end{figure}
Rates of neutrons 

For a complete understanding of the low energy ($<$1 MeV) background in the BDX-Hodo crystal, particles produced in the dump can not tracked separately since some of them are produced along the way (e.g by energetic muons or neutrons in the proximity of the detector).
Moreover  neutral particles (in particular low energy/thermal neutron) do not directly interact with the crystals but deposit a visible energy via secondary interactions (e.g. gamma from nuclear capture in the surrounding material) making hard, if not impossible,  to track back the background source. For all the above mentioned reasons we evaluated the background running the full FLUKA simulation of 11 GeV electrons interacting with the beam-dump and looking at the spectrum of deposited energy in BDX-Hodo (the crystal) positioned  in the three locations of interest. This , unfortunately, require to accumulate a sizeable statistic since no bias weight optimization is possible. Based on the (limited) statistics generated so far (corresponding to 4.4 10$^7$ EOT or, equivalently 7 10$^{-6}$ $\mu$A $\times$  s) we only see hits in location {\bf A} corresponding to an extrapolated flux of $\sim$ 1 MHz /$\mu$A .
Since test will run with BDX-Hodo positioned in   {\bf B} or {\bf C} we expect the crystal to be exposed to  a lower background rate (of the order of 100kHz in   {\bf B}  and 10 kHz in {\bf C} ). Even a background rate in the range of 1 MHz does not represent an issue since the low energy deposited ($<$1 MeV) corresponds to signals of one/few photoelectrons spread over the entire scintillation time window ($\sim 1 \mu$s) making it not distinguishable from the other noises (e.g. SiPM dark current, preamplifier noise, ...).








\subsubsection{Cosmic background}
The cosmic muon background in the BDX-Hodo has been evaluated using GEMC. This is the same cosmic flux generator used in PR-16-001~\cite{bdx-proposal}. The muon spectrum has been divided in different ranges  and correctly weighted to estimate the full rate expected on the detector. Rates in the detector have been evaluated for CsI(Tl) crystal alone, Top scintillator alone (showing the maximum rate) and requiring  the coincidence of the front/back scintillator with the crystal (the condition used to identify and count muons produced in the beam-dump). Detection thresholds were set to  10 phe and 100 phe for scintillators and CsI. Tab.~\ref{tab:cosmic} shows the results of this study. The cosmic rate is negligible (in every condition $<$ 1 Hz) well below the expected rate of muons from the beam-dump.

\begin{table}[htp]
\caption{Cosmic rate expected in different components of BDX-Hodo}
\begin{center}
\begin{tabular}{|c|c|c|c|}
\hline
Energy range  (GeV) & Rate$_{Crystal}$  (Hz)&  Rate$_{Top\,Scintillator} $(Hz) & Rate$_{Coincidence}$ (Hz) \\
\hline\hline
 0.2 - 2  & 0.01 &  0.02 & 0\\
 \hline
 2 - 10  & 0.2 &  0.25 & 0.01\\
 \hline
 10 - 100  & 0.35 &  0.4 & 0.01\\
\hline\hline
Cosmic muon rate  & 0.56 &  0.67 & 0.02 \\
\hline\hline
\end{tabular}
\end{center}
\label{tab:cosmic}
\end{table}%

\subsection{Test configuration and practical details}
Practical details (drilling technology, costs and schedule) and a work plan for the proposed test configuration  are reported in the Appendix.

\subsection{Summary}
We simulated the interaction of a 11 GeV electron beam with Hall-A beam-dump studying the expected  radiation field in the beam-dump vault and in the downstream area ($\sim$20 m away) where shall be located the new underground  facility required by the BDX experiment. Two different  simulation tools (GEMC and FLUKA) where used. For some locations, results were compared with JLab Radiological Control Group estimates. Here are our main findings:
\begin{itemize}
\item{our results are consistent with what obtained by RadCon;}
\item{we confirm that only (high energy) muons and (mainly  thermal/low energy) neutrons propagates trough the beam-dump vault concrete walls  reaching the region of intereset; }
\item{no sizeable flux was found outdoor above the ground   in the proximity of the beam-dump;}
\item{for energy greater than 100 MeV muon and neutron flux estimated with FLUKA and GEMC well match;}
\item{for energy lower than 100 MeV  FLUKA (using the biasing technique)  resulted more efficient in run-time. }
\end{itemize}
To validate MC tools and gain confidence in the beam-on background shielding optimization  for the  BDX experiment we propose to measure the muon flux in the region where the new underground facility will be located. 
Here below is the proposed experimental set up and the expected results:
\begin{itemize}
\item{muons produced in the dump can be accessed by playcing a downstream detector intercepting the beam line continuation; }
\item{a detector  (BDX-Hodo) based on one CsI(Tl) crystal from  BDX ECal, sandwiched between layers of scintillator counters  will be specifically built for this measurement;}
\item{ two wells equipped by  10' pipes will be drilled in two positions and the BDX-Hodo detector downed  till to reach the beam axes continuation;}
\item{rates of beam-on muons measured by BDX-Hodo are expected  to be sizeable ( $\sim$ 0.1 kHz - 50 kHz) for a wide intervals of the Hall-A beam current (1-100 $\mu A$) making the test fully parasitic wrt the Hall-A plans;}
\item{this measurement was found to be insensitive to the cosmic muon background and other backgrounds (mainly) neutrons generated in the dump; }
\item{the use of a BDX ECal  crystal will allow to prove the proposed technology in a background-rich environment (the BDX experiment calls for an optimised shilding that will drastically reduce  any possible background); }
\item{once the pipes will be inserted, tests will run for $\sim$ a week, in parallel and parasitically  wit respect to  any 11 GeV 1-100 $\mu$ A, Hall-A run;} 
This  test, measuring the muon flux (absolute and relative) in different location in Z (distance from the dump) and Y (vertical) will address the concern expressed by PAC44 report about the bema-on background in the BDX experiment.



