\documentclass[11pt]{report}

\textwidth  6.5in
\textheight 8.0in
\topmargin 0. in
\oddsidemargin -0.0in
\evensidemargin 0.0in
\pagenumbering{arabic}
\linespread{1}

%\usepackage{graphics}
\usepackage[dvipdf]{graphicx}
%\usepackage{subfig}  % For subfloats
\usepackage[final]{pdfpages}
\usepackage{subfigure}  % For subfloats
\usepackage{color}
\usepackage{multirow}
\usepackage{epsfig}
\usepackage{wrapfig}
\usepackage{rotating}
\usepackage{amsmath}
\usepackage{footmisc}
\usepackage{titlesec}
\usepackage{wasysym}
%\usepackage{subfigure}
\usepackage{hyperref}
\usepackage{float}
\newcommand{\footlabel}[2]{%
    \addtocounter{footnote}{1}
    \footnotetext[\thefootnote]{%
        \addtocounter{footnote}{-1}%
        \refstepcounter{footnote}\label{#1}%
        #2%
    }%
    $^{\ref{#1}}$%
}
\usepackage{setspace} 
\usepackage{rotating}



\usepackage{draftwatermark}
\SetWatermarkLightness{0.8}
\SetWatermarkScale{5}






\begin{document}


\title{BDX: on-beam background assess and MC validation}

\author{
M.~Battaglieri,
A.~Celentano,
R.~De~Vita,\\
\it \small Istituto Nazionale di Fisica Nucleare, Sezione di Genova, Genova, Italy\\ \\
M.~Carpinelli\\
\it \small  Istituto Nazionale di Fisica Nucleare, Sezione di Pisa e Dipartimento di Fisica dell'Universit\`a, Sassari, Italy\\ \\
A.~D'~Angelo, A.~Rizzo\\
\it  \small Istituto Nazionale di Fisica Nucleare, Sezione di Roma-TorVergata e Dipartimento di Fisica dell'Universit\`a, Roma, Italy\\ \\
N.~Randazzo\\
\it  \small Istituto Nazionale di Fisica Nucleare, Sezione di Castania e Dipartimento di Fisica dell'Universit\`a, Catania, Italy\\ \\
}


\date{\today}
\maketitle

\begin{abstract}

This proposal presents the Heavy Photo Search (HPS) experiment planned at Jefferson Lab for 2014-15 and a possible involvement of the INFN-JLAB12 Collaboration. This document is based on the 
HPS proposal to Jefferson Lab PAC40 submitted at the end of May 2013. 
INFN-JLAB12 groups are expected to mainly contribute to the electromagnetic calorimeter ECal, the key-element of the HPS trigger system. 
The expected activities are: the motherboard design, manufacture and test; 
the  Light Monitoring System (LMS) design;  the replacement of the existing APDs with  new  Large Area APDs.
The INFN-JLAB12 commitment is expected to be both in term of funding (LAAPDs procurement, motherboard manufacturing, LMS prototyping),
 and manpower (motherboard design and test,  LMS design and prototyping,  photo-sensor replacement).
New Italian groups , specifically focused on HPS experiment, are joining the INFN-JLAB12 Collaboration
providing the necessary manpower to accomplish the proposed activities.
The detailed funding request to the INFN-CSNIII will be placed in June/September 2013. 

\end{abstract}

\newpage

\tableofcontents\newpage


\section{The HPS experiment}
The Heavy Photon Search (HPS) is an experiment proposed for Jefferson Laboratory to search for new heavy vector boson(s), aka "heavy photons" or "dark photons" or "hidden sector photons", in the mass range of $20$ MeV/c$^2$ to $1000$ MeV/c$^2$. Such particles will arise if there are additional U(1) gauge bosons in nature, and they will couple, albeit weakly, to electric charge through kinetic mixing. Many BSM theories predict the existence of additional U(1)'s, and recent observations of high energy electrons and positrons in the cosmic rays may be the result of primordial dark matter annihilating into heavy photons. HPS searches for  electro-produced heavy photons using both invariant mass and separated decay vertex signatures using a compact, large acceptance forward spectrometer. 
The first stage of HPS, the HPS Test Run, was approved by the Jefferson Lab PAC37 on January 14, 2011, after which it was proposed to DOE HEP for funding and approved and funded by Summer 2011. The Test Run was built in 2011-2012, and installed and run at JLAB in Spring, 2012.  PAC39 reviewed  HPS in June, 2012, and on the basis of the successful run, granted it an "A" rating, a commissioning run with electron beams, and approval (C1) to proceed to the full experiment contingent only on final approval from JLAB management. The full HPS experiment, presented at PAC40 in June 2013,  is capable of searching for heavy photons over a wide and unchartered region in parameter space. Schedules at Jefferson Laboratory admit time for HPS to be commissioned in Hall B in the Fall of 2014 and to take data beginning in Spring 2015. With final approval from JLAB management, and with timely DOE approval and funding, the HPS Collaboration can design, build, test, and commission HPS in time to take advantage of these scheduling opportunities, and begin in earnest its search for spectacular new physics at the intensity frontier.   

\section{The HPS Collaboration}
The HPS Collaboration whose spokesmen are J.Jaros (SLAC), S.Stepanyan (JLab), and M.Holtrop (University of New Hampshire), includes about sixty people from about twenty different institutions from all around the world.
In Italy, the HPS experiment is part of the INFN-CSNIII JLAB12 experiment. The participating INFN groups,  and related Universities, are:  Cagliari, Catania, Genova, Sassari, Roma Tor Vergata.
According to the {\it HPS Collaboration Charter and Bylaws} to become member of the HPS Collaboration each of the participants need to commit him/her-self to contribute to the experiment in instrumentation and/or software.
The commitment statement reported in this document is intended to respond to this requirement, providing the HPS Collaboration membership to all the authors. The document will be submitted to the HPS Collaboration Executive Committee
for the necessary approval. Each INFN group will define a PI  responsible for informing the HPS executive Committee of any changes within his group pertaining the membership.
The Italian contribution and commitment will be coordinated  with the relevant HPS Working Group.

\begin{figure*}[t]
\includegraphics[width=\textwidth]{HPS-setup.pdf}
\caption{\small{The HPS detector components. From left to right: the target, the silicon tracker in the dipole magnet, the electromagnetic calorimeter ECal and the muon detecor.}}\label{fig:hps-setup}
\end{figure*}

\section{Experimental set-up}
HPS will utilize a setup located at the upstream end of  the experimental Hall-B at Jefferson Lab. The setup will be based on a three-magnet chicane, the second dipole magnet serving as the analyzing magnet for our forward spectrometer. The detector package, shown in Fig.~\ref{fig:hps-setup}, will include a silicon tracker, an electromagnetic calorimeter, and a muon detector. High luminosities are needed to search for heavy photons with small couplings and masses in the $20$ to $1000$ MeV range. Utilizing CEBAF essentially continuous duty cycle, the experiment can simultaneously maximize luminosity and minimize backgrounds by employing detectors with fast response and rapid readout. The HPS setup is designed to run with $> 200$ nA electron beams at energies from $1.1$ GeV to $6.6$ GeV impinging on a tungsten target of up to 0.0025 $X_{0}$ located  10 cm upstream of the first layer of the tracker.

The HPS tracker consists of six double layer planes, 36 microstrip sensors in total. Placing the planes of the tracker in close proximity to the target means that the primary beam must pass directly through the middle of the tracking detector. This has necessitated that the sensors don't encroach on a “dead zone”, where multiple Coulomb scattered beam particles and radiative secondaries are bent into the horizontal plane, the so-called "wall of flame".  However, since the energy released in the decay of a low mass A$^\prime$ is small relative to its boost, the opening angle between decay daughters can be quite small. To maximize the acceptance for low mass A$^\prime$s, the vertical extent of the dead zone must be minimized and sensors placed as close as possible to the beam, so our design incorporates precision movers that can bring the silicon detectors  to the required positions. Since interactions of the primary beam with air or even helium at atmospheric pressure gives rise to low-momentum secondaries that generate unacceptable occupancies, we have chosen to place the entire tracking and vertexing system in vacuum, in the Hall B pair spectrometer's magnet vacuum chamber. Silicon microstrip sensors are used in the tracker/vertexer because they collect ionization in $10$’s of nanoseconds and produce pulses as short as $50-100$ ns. The sensors are read out continuously at $40$ MHz using the APV25 chip, developed for the CMS experiment at the LHC. 

A lead-tungstate with APD readout  electromagnetic calorimeter is used to trigger the DAQ. Like the tracker system, the electromagnetic calorimeter is split to avoid impinging on the “dead zone”. The beam and radiative secondaries pass between the upper and lower ECal modules, which are housed in temperature-controlled enclosures, needed to stabilize the energy calibration. The ECal is the HPS detector component which the INFN-JLAB12 Collaboration  wants to contribute to: a more detailed description is  reported in the next Section.

The muon detection system will be installed behind the ECal, which has absorbed most of the electromagnetic background produced in the target. The remaining backgrounds will be attenuated by the first absorber layer of the muon system. The muon system will consist of four double layers of scintillator hodoscopes sandwiched between iron absorbers. Light from scintillator strips will be transported to photo-multiplier tubes via wave-length shifting fibers embedded within the strips. As in case of the ECal, muon system will be divided into two parts, beam up and beam down. There is a vacuum chamber between the two parts to allow the beam and radiated secondaries to be transported in vacuum.




\section{The electromagnetic calorimeter ECal}
The Ecal, depicted in Figure \ref{fig:ecal}, consists of $442$ lead-tungstate PbWO$_4$ crystals with avalanche photodiode (APD) 
readout and amplifiers. Indeed,(PbWO$_4$) crystals with APD readout 
are ideally suited to deal with the expected high radiation and high rate environment and they can operate in the fringe field of the HPS
magnetic field as well.
The lead-tungstate modules, see Figure \ref{fig:module}, are taken from the Inner 
Calorimeter (IC) of the JLab CLAS detector, which was built by IPN Orsay (France) and other Hall B collaborators and was used
in a series of high energy electroproduction experiments. Orsay has the responsibility of ECal mechanics and preamplifiers. 
The Ecal data is digitized in the JLAB FADC250, a $250$ MHz flash ADC developed for the JLAB $12$ GeV Upgrade program.  
The full analogue information from the FADCs coupled with the fine spatial information of the calorimeter is available to the trigger, which uses energy deposition, position, timing, and energy-position correlations to reduce the trigger rate to a manageable $\sim 30$ kHz. 
 
The PbWO$_4$ crystals are $16$ cm long and tapered. The cross section of the front face is $1.3\times 1.3$ cm$^2$, 
and  $1.6\times 1.6$ cm$^2$ at the back. Modules in the ECal are arranged in two formations, as shown in Figure \ref{fig:ecal}. 
There are 5 layers in each formation; four layers have $46$ crystals and one has $37$. The ECal is mounted downstream of the 
analyzing dipole magnet at the distance of about $137$ cm from the upstream edge of the magnet.
 In order to stabilize the calorimeter's performance, 
the crystals, APDs, and amplifiers are enclosed within a temperature controlled environment, held constant at 
the level of 1\!\char23F. The energy resolution of the system, expected from operational experience with the IC, 
is $\sigma_E/E \sim 4.5\%/\sqrt{E}$ (GeV). As in the IC, PbWO$_4$ modules are connected to a motherboard that provides
power and transmits signals from individual APDs and amplifier boards to the data acquisition system. 
\begin{figure*}[t]
\includegraphics[width=\textwidth]{ECal.png}
\caption{\small{Arrangement of Ecal crystals. The two modules are positioned above and below the beam plane. Each module has 5 layers. 
There are 46 crystals in each layer, with the exception of the layers closest to the beam plane in which 9 crystals are removed to allow 
a larger opening for the outgoing electron and photon beams.}}\label{fig:ecal}
\end{figure*}

\begin{figure*}[t]
\includegraphics[width=\textwidth]{ecal_module.png}
\caption{\small{The ECal module is composed of a 16 cm long lead-tungstate crystal, Avalanche Photo Diode, and a amplifier 
board.}}\label{fig:module}
\end{figure*}

The HPS calorimeter was built and used during the HPS test run in April-May of 2012. This was the first time that 
a readout and trigger system utilising the JLAB FADC250s had been used in a real experiment. 
While the ECal performance during the test run was satisfactory, there are some aspects that can be improved by the INFN-JLAB12 Collaboration contribution.

\begin{figure*}[t]
\begin{center}
\includegraphics[scale=0.4]{hps-motherboard.pdf}
\caption{\small{The new HPS motherboard conceptual design.}}\label{fig:new_mb}
\end{center}
\end{figure*}
\section{The INFN-JLAB12 commitments}

\begin{enumerate}
    
\item {\bf Mmtherboards modification} - One of the issues  faced during the test run was excess noise on some channels of the motherboards. 
Most missing channels were in fact switched off because they were very noisy and there was no time to debug them. 
New motherboards will be designed and built to resolve these noise issues. One of the options under discussion is to replace long motherboards with shorter ones with power and signal connectors 
located on the top (for the top module) and the bottom (for the bottom module) of the thermal enclosure. 
Figure~\ref{fig:new_mb} shows the conceptual design of the new HPS motherboard. Each motherboard connects up to 60 APD. For the whole ECal readout 6+2 identical motherboards are foreseen. 
Two of them have a reduced number of channels and a different geometry to match the central beam-hole. APDs will be HV-supplied in group of 10 to keep a reasonable gain-matching. 
HV and signal connectors will be placed in the upper/bottom part requiring a modification of the mechanical enclosure. The Orsay collaborators will take care of the necessary mechanical modifications.
The motherboard design takes advantage of the experience gained by the INFN-JLAB12 groups in building the FT-Cal and the FT-Cal Prototype motherboards.
In particular, specific precautions to reduce noise pick-up by signal tracks and reduce the cross-talk will be taken. These includes: multiple-grounds planes in the PCB and separation of adjacent signal tracks with grounds.
The design will be done by the INFN-GE group supervised by the Electronic group of Orsay that originally designed the IC motherboard.
\item {\bf  Photosensor replacement} - Installing new APDs on the current crystals will significantly improve the ECal performance. Replacing the old $5\times 5$ mm$^2$ 
Hamamatsu S8664-55 APDs with $10\times 10$ mm$^2$, Hamamatsu S8664-1010 (LAAPD) will improve two critical characteristics of the present calorimeter modules. 
First, the new APDs from Hamamatsu have much better performance than the ones which are currently installed. In particular, being made by the same wafer, they 
show an excellent gain uniformity. For a chosen working point of $G=150$, a $\Delta G=\pm 10\%$ corresponds to a supply bias variation of  $\sim 1.5$ V.
For the current APDs the difference is now in the range of 50V. A reduced variation allows to supply the new LAAPDs in group of 10 keeping the gain variation within 
a minimum value. In this way it will be possible to achieve much better 
uniformity in the response of the calorimeter modules, and consequently better uniformity in trigger thresholds. 
Secondly, the new APDs have a $4$ times larger readout area, ensuring $4$ times more light collection and therefore $4$ times larger signals. 
This will allow the use of lower gain amplifier modules which in turn will decrease electronic noise. Tests 
performed for the FT-Cal, showed that amplifier boards 
with a factor 2 lower gains have a noise level $<5$ MeV. The energy deposition in the HPS PbWO$_4$ crystals  
from minimum ionizing cosmic muons passing transversely to the crystal axis is $\sim 15$ MeV. Moving the energy thresholds 
close to $5$ MeV will allow the MIP peak to be clearly distinguished, and will let the calorimeter  be calibrated and monitored with cosmic muons. 
We performed in Genova the first tests of the Hamamatsu $10\times 10$ mm$^2$ LAAPDs and a new amplifier board on HPS crystals. In 
Figure \ref{fig:mip10x10}, the charge distribution of a single crystal is shown for $5\times 5$ mm$^2$ (left) and $10\times 
10$ mm$^2$ (right) APDs. A coincidence signal from scintillator telescopes positioned above and below the module provided the trigger. 
The crystal was positioned horizontally as it would be in HPS, so the cosmic muons would pass through it perpendicular to the crystal axis. 
The red line histogram is for all events triggered by the scintillation telescope and corresponds to the noise. The black line histogram corresponds 
to the charge detected within $100$ ns. The MIP peak is clearly visible and well isolated from the noise when the S8664-1010 LAAPD 
is used. For  the S8664-55 APD, the MIP signal is also seen, but its charge distribution is under the noise peak 
and the peak position can not be resolved. MIP calibration, together with the light monitoring system, will ensure stable and reliable performance of the ECal and the trigger system. 
As a bonus, the lower noise will allow the use of lower  thresholds and improve the calorimeter's energy resolution.
The increased APD size  will increase the capacitance (by almost a factor of 4) requiring a new preamplifier with an adapted input impedance and a lower gain.
The new preamplifier, whose design was optimised for the FT-Cal detector will be built by the Electronic Group of Orsay {\it and funded by the HEP-DOE grant}.
\begin{figure*}[t]
\includegraphics[scale=0.37]{MIP_5x5_APD.png}
\includegraphics[scale=0.37]{MIP_10x10_APD.png}
\caption{\small{Charge distribution from readout of the HPS calorimeter crystal with Hamamatsu S8664-55 (left) and S8664-1010 
(right) APDs, and the new low noise amplifier board. The red line histogram corresponds to the charge distribution for all triggers 
comming from the scintillators positioned above and below the crystal. The black line shows the distribution for hits in the crystal 
within $100$ ns of the trigger signal. }}\label{fig:mip10x10}
\end{figure*}
\item {\bf Light Monitoring System (LMS)} - For an experiment like HPS, where backgrounds must be well understood and need to be strongly suppressed, the trigger bias must be minimized. In particular, having stable, known, and uniform thresholds at the trigger readout is necessary in order to avoid  bias in the 
event selection. Such uniformity and stability can be achieved with the installation of an online gain monitoring 
system. This system will introduce short light pulses into the front face of the crystals. The crystals already have fiber holders attached, allowing implementation of this system without modification of  the crystals or the assembly. 
Optical fibers will be used to transmit light from a calibration  source to the crystals to test the response of the APDs. The response of the system could change in time because of 
losses in crystal transparency due to radiation damage or because of gain variations of the APDs. 
Such a calibration system has been developed for several experiments (PHOS and CMS at CERN for instance) with various light sources. The system for the ECal 
will be similar to the monitoring system developed by INFN-GE for the FT-Cal, based on fast LED pulser coupled in air or with a optical fiber to the front of the crystal.
Each module will have a red and blue mono-color LED light source for monitoring purposes. 
Blue light transmission, corresponding to the domain of the crystal's emission spectrum, is very sensitive to the presence of color centres which are produced by radiation damage. So the blue light source will monitor variations of the response in the main domain of the spectrum.
The response to red light is less sensitive to the color centers,  and so permits monitoring the APD gains more directly. Thus the use of two colors separates gain variations due to the 
APD and electronics from those due to changes in the crystal transparency, and provides clear information on the state of the electronics. 
The FT-Cal monitoring system, with a single blue LED, directly coupled to the front of the crystal has been designed in Genova and used in the on-beam tests of the  FT-Cal prototype.
The system shows excellent performance in terms of dynamic range (from an equivalent energy deposition of 10 MeV to few GeV per crystal), timing (down to $\sim$200ps)  and stability (the ratio of two different channels show a stability of few 
$\permil$ over a week). It is not yet defined which solution will be adopted: direct LED coupling or coupling via an optical fiber. In both cases,
thanks to the modular design of the FT-Cal LMS, the modifications required to match the HPS specifications is minimal.
 The design of the LMS will be performed in conjunction with the Orsay group, responsible for the subsequent production.
\end{enumerate}

\section{Time schedule}
\subsection{Run plan and beam time request}
The Jefferson Laboratory PAC39 graded HPS physics with an ``A,''   approved a commissioning run with electrons, and granted the approval 
for the full HPS experiment conditioned to successful completion of the Test run.

The total beam time requested in our original proposal to PAC37 is 180 days.
Anticipating early running in Hall-B, we propose to conduct HPS in two phases. The first phase, expected to run 
in 2014-2015, will complete the commissioning run and begin the production running. For the upcoming commissioning run in 2014 
we have requested 3 weeks; for the data taking runs in
2015, we have requested 5 weeks. The second phase of HPS, which will use the remaining beam time, can be scheduled later in 2015 
or in 2016 and beyond, and will continue runs at 2.2 and 6.6 GeV and possibly other energies.
If more beam time is available in 2015, HPS will continue data taking at these or other energies.

An addendum to the HPS proposal, reporting results of the test run,  has been submitted to JLab PAC 40, to be held in June 2013, to get the final approval and have the beam time assigned according  to the schedule
presented above. PAC results will be available in late June 2013. 


\subsection{HPS schedule}
Our goal is to be ready to install HPS in Hall-B by September 2014, and proceed with commissioning on beam with the 
CEBAF early physics beam window opportunity in October 2014. The data taking will continue until summer 2015. Meeting this schedule
will require approval and funding as soon as possible, preferably by June 2013. Schedules for each of the major subsystems of the 
experiment are  summarized here. The total construction schedule extends over 16 months, assuming the funding will be
available from mid-2013. The schedule  includes a contingency of about 20\%. 

The conceptual design of the beam-line will be done during 2013. 
Final Engineering and Construction will start in Spring 2014 and be completed well before the installation time in October 2014, 
providing substantial float. 

The Test Run SVT was shipped back to SLAC by early February 2013 to 
rework the modules for the first three layers of the HPS and commission the motion control systems. The conceptual design of the 
Layers 1-2-3 and Layers 4-5-6 have started already in spring 2013. 
The SVT DAQ will formally begin work in  the second half 2013.
 The assembly and integration test at SLAC are expected in spring 2014 and the SVT will 
be ready for shipping on July 2014. The SVT will be ready for installation in mid-August 2014. SVT  installation in the analyzing 
magnet vacuum chamber will occur in September, depending on the schedule of the Hall-B
\subsection{ECal schedule}
The Ecal work will start in the second half 2013 and run through August 2014. The ECAL will be ready for installation by September 2014.
The calorimeter will be disassembled and the crystals sent to Italy for the APD replacement in fall 2013.
In late summer - fall 2013 the new motherboards will be designed, produced and tested to be ready before the end of the year. 
The new preamplifiers are expected to be ready at the same time. Electric and electronics tests will be performed before the end of the year.
The mechanics for the new mounting system will be manufactured in early spring 2014. The crystals with the new LAAPDs are expected to be sent to JLab for July 2014 and the final 
ECal assembly is expected in August 2014. At the beginning of  September  the ECal should be ready for the installation in the Hall.
\subsection{INFN activity schedule}
The main INFN contributions to HPS are in the following areas: motherboard design, production and test, light monitoring system design and photo-sensors replacement.
 The first two activities will start in early fall to be completed before the end of the year. The replacement of the APD with the new LAAPD is a more complex task since it requires a 
tight coordination between the funding agency (INFN-CSNIII) and the vendor (Hamamatsu) to be sure that  we will keep tightly on schedule.
LAAPD funding is expected in October 2013 and the order of 500 LAAPDs (442 channels + 58 spares)  will be split in two parts: the first lot will be ordered within 2013, while the
remaining part will be ordered in January 2014 for budget reasons. LAAPDS will be tested and benchmarked to establish the working condition, gain as a function of bias voltage and temperature,
upon delivery. Tests are expected to happen in  March - April 2014. During fall/winter 2013 all tools necessary for crystal machining will be prepared in advance and provided to the different work-sites.
 Crystals will be shipped from JLab to Italy well in advance to let the involved groups to disassemble
(crystal + APD + wrapping + plastic nose) and clean the crystals,  while  waiting for the new LAAPDs. The glueing and re-assembly procedure will start as soon as the LAAPD will be delivered and tested,  
expected  to happen in April - June 2014. A final test and characterisation of  each crystal will be performed at the end of the machining,  before shipping back the crystals to JLab.
The crystal will be available on-site in August/September 2014 to allow the ECal assembling and installation in the Hall before the beginning of the run. A detailed timetable of the different
 tasks is reported in Tab~\ref{tb:infn_ts}.
\begin{table}[htdp]
\caption{INFN-JLAB12 tasks commitment and time-schedule.}
\begin{center}
\begin{tabular}{|c|c|c|}
\hline
Activity & Place  & Time\\
\hline\hline
New motherboards design & GE &June - September 2013\\
\hline
Iterations with mechanical engineers & GE Orsay &June - August 2013\\
\hline

New motherboards production and test & GE & October - December 2013\\
\hline 
Light monitoring system design & GE & September - December 2013\\
\hline
Tooling for crystal assembly (glueing) & GE	& October - December 2013\\
\hline
Tooling for LAAPD test & GE&October - November 2013 \\
\hline
Crystal collection from JLab &CA-SS CT GE&  October - November 2013 \\
\hline
Old APD detachment & CA-SS CT GE&December - February 2013-14  \\
\hline
LAAPD procurement &  & October - February 2013-14  \\
\hline
LAAPD test & RM2 &  March - April 2014  \\
\hline
Crystal assembly &CA-SS CT GE  & April - June 2014  \\
\hline
Crystal test and benchmarking &CA-SS CT GE  & June - July 2014  \\
\hline
Crystals shipping to JLab & JLab & July 2014  \\
\hline
ECal assembly & JLab & August - September 2014  \\
\hline
\hline
\end{tabular}
\end{center}
\label{tb:infn_ts}
\end{table}%

\section{Cost and funding}
\subsection{HPS cost and funding}
The total cost for HPS is \$2.971 M, consisting of \$1.796 M capital equipment, \$927 K operations, and \$248 K infrastructure. 
HPS is seeking funding from other sources for the Muon System and upgrades to the Ecal.
The College of William\&Mary (Williamsburg, VA) has submitted an MRI proposal to NSF for the Muon System, requesting $\sim \$200$k. IPN ORSAY (France) 
has submitted a proposal to a French funding agency for the various ECal upgrades, including an ECal Light Monitoring System (\$100k),
 and other expenses related to ECal fabrication and test.

\begin{figure*}[t]
\begin{center}
\includegraphics[scale=0.8]{hama_quote.pdf}
\caption{\small{Hamamtsu quote }}\label{fig:hama_quote}
\end{center}
\end{figure*}

\subsection{INFN-JLAB12 commitment}
The budget requested to the INFN-CSNIII is reported in Tab.~\ref{tb:infn_funds}. The main cost  is related to the LAAPD procurement (see the attached quote of 215k euro). The other items (motherboards, LLMSMS prototyping, tooling and shipping) 
account for a total of 35k euro. The precise  sharing among the INFN-JLAB12 groups will be presented to the INFN-CSNIII in June/September 2013. 
 It worth to be noticed that, since the final destination of the machined crystals is JLab (US), the LAAPD purchase is VAT exempted.
The total request of about 280k euro already takes it into account.


\begin{table}[htdp]
\caption{INFN-JLAB12 funding contributions.}
\begin{center}
\begin{tabular}{|c|c|}
\hline
Component& Cost (Euro) \\
\hline\hline
LAAPDs (500pcs Hamamatsu S8664-1010 Sel3) & 215k \\
\hline 
Motherboards & 5k \\
\hline
LMS prototype & 4k \\
\hline
Tools for crystal assembly & 5k \\
\hline
Tools for LAAPD tests & 5k \\
\hline
Shipping & 8k \\
\hline
Consumables & 8k \\
\hline
\hline
Total + VAT + 10\% contingency & 283k\\
\hline
\end{tabular}
\end{center}
\label{tb:infn_funds}
\end{table}%
\section{Manpower}
\subsection{HPS manpower}
The manpower needed to design, fabricate, assemble, test, install, and commission the HPS is summarized below.
The HPS Collaboration successfully mounted the HPS Test Run experiment on a very aggressive schedule, and has the personnel needed to realize full HPS.

Beam-line conceptual  and mechanical design will be done at JLab with contributions from the Engineering staff at SLAC.
The Tracker/Vertexer is being designed and engineered by SLAC as well as 
the SVT DAQ. The Ecal work is being coordinated by the Orsay Group 
with participation by the JLab group.  
The Ecal Trigger/DAQ work is done in JLAB DAQ group, which supports Hall B activities at  and with JLAB Fast Electronic group, which has designed the FADC250.
JLab will take care of  assembling and testing the electronics, programming the trigger, and integrating it with 
the Ecal hardware. Slow control programming is also being done by JLAB personnel.
Simulation work is supported by New Hampshire, SLAC and JLAB,  Data management and storage and computing infrastructure will be overseen by JLAB Computer Center.
The HPS collaboration is about 60 strong, so has adequate manpower for overall installation, commissioning, and data taking. 

\subsection{INFN-JLAB12 manpower and resposabilities}
A detailed work-plan has been established  with a realistic estimate of the necessary manpower for the different tasks. Here below is reported the detailed list.
\begin{enumerate}
\item {\bf Modification of motherboards}  - 
The motherboard design requires 30 days of FTE engineer to define the electrical  layout, lines tracing, place components on the PCB  and validate the project,
plus several iterations with the Orsay HPS mechanical engineer to define the  modifications to the enclosure  box, since the new design moves the gap where the PCB feeds-trough from the lateral sides to the top/bottom sides.
After receiving the motherboard from the vendor, 15 days of FTE engineer are necessary for testing the electric connections,  the cross-talk and test the motherboard functionality  by some samples (full crystals+LAAPD+preamps chain).   
No  tests in Italy  of the motherboard with the whole  crystal set  mounted are foreseen due to the leak of a convenient mechanical framework. These tests will be done at JLab by on-site collaborators.
This task will be performed in Genova.
\item {\bf  Photosensor replacement} - 
This task will be performed in Roma-Tor Vergata, Genova, Catania and Cagliari. The procedure can be split in four phases:
\begin{itemize}
\item LAAPD benchmarking: the 500 detectors needs to be characterise to measure the gain of each sensor as a function of the applied bias voltage and temperature, $G(V,T)$. The measurement will be performed using 
the same benchmarking system used to test the FT-Cal LAAPDs. Two twin systems allow to test up to 24 APD per day (about 4 hours per each T set-point)  requiring a total of 40 days of  FTE technical collaborator to characterise the whole batch.  
\item Crystal preparation: the PbWO crystals need to be carefully  de-wrapped from the VM2000 wrapping envelope, placed in alcohol for a whole night  to weak or dissolve the glue between the crystal and the APD, remove the APD 
 and  clean the crystal face by acetone. To perform this operation two arrays of gas-tight plastic containers, partially filled by alcohol, will be prepared in advance. The estimated time necessary for tooling is 5 days FTE technical collaborator, the same for optimise the cleaning procedure (both done in Genova); 15 days + 15 days of FTE technical collaborators for the whole job (assuming two independent work-sites, Cagliari and Catania).
\item LAAPD glueing: having the crystal cleaned the next phase is to glue the LAAPD and place the old wrapping around. The glueing procedure takes a full night to harden the glue and therefore the job needs to be done in parallel over a set of many crystals at time.  Two  "L"-shape aluminium+neoprene racks  to keep the crystals in vertical position will be prepared in advance. The time necessary to perform this operation is the same as for the cleaning phase and can be shared between two sites.
\item Assembly test: in the last phase the full assembly will be tested to check, within 20-30$\%$ accuracy, if the glueing and wrapping procedure was correctly performed. About 20 crystals will be placed over a fixed-temperature plate in a T-controlled room and let wait for a night to reduce any T-related effect (APD-Gain dependence, LED emission,~...). Each crystal will be tested by connecting one reference LED pulser from one side and a preamplifier to the LAAPD. Each data taking will take 10 minutes to accumulate enough statistics. When done, the LED and the preamp will be moved to the next crystal. The data acquisition can be performed with an oscilloscope able to accumulate the signal amplitudes in a histogram. 
 We estimate 15 days of FTE to design and produce the necessary tool, 5 days to test and optimize the benchmarking procedure (both done in Genova)  and then 15+15 days to perform the test (assuming two independent work-sites, Cagliari and Catania).
\end{itemize}
For crystals showing a poor light transmission, out of the desired range, the whole cleaning/glueing/testing procedure will be repeated. We  consider a failure fraction of about $10\%$ adding a safety-factor of 5+5 days.

\item {\bf Light Monitoring System (LMS)} - We can adapt the light monitoring system developed for the FT-Cal, that thanks to its modular design can be widely reused and adapted to the HPS ECal geometry with a minimal effort.
 The system is split in three parts: the controller board, the drivers board and the LED board. 
The controller does not need any change while the drivers board layout need to be redesigned, as well as the LED PCB.  If, beside the blue LED, a red LED will be used, a selection of the model available on the market is needed. 
We foresee 30 days of FTE electronic engineer to design and test a prototype of the LMS. 
The LMS design and  prototyping will be done in Genova. 
\end{enumerate}
To summarize, INFN-GE needs to allocate: 45 days of FTE Electronic engineer for the motherboards design and production,  30 days of FTE Electronic engineer for the Light Monitoring System design and 35 days FTE of technical collaborator for the APD-related work; both INFN-CT and INFN-CA needs to allocate 50 days of FTE technical collaborators (sharable between two technicians); INFN-RM2 needs to allocate 40 days of FTE technical collaborator (sharable among two technicians).

\begin{table}[htdp]
\caption{INFN-JLAB12 manpower commitment.}
\begin{center}
\begin{tabular}{|c|c|c|}
\hline
Activity & INFN-JLAB12
 manpower (FTE) & Site\\
\hline\hline
New motherboards design & 30 days & Genova\\
\hline
Individual motherboards tests & 15 days  & Genova\\
\hline
Full motherboard assembly tests &  & JLAB\\
\hline
\hline
LAAPD characterization& 40 days  & TorVergata\\
\hline
\hline
Crystal preparation: tooling & 5 days  & Genova\\
\hline
Crystal preparation (50 pcs): procedure optimization & 5 days  & Genova\\
\hline
Crystal preparation (200 pcs): un-gluing and cleaning & 15 days  & Catania\\
\hline
Crystal preparation (200 pcs): un-gluing and cleaning & 15 days  & Cagliari\\
\hline
LAAPD glueing: tooling & 5 days  & Genova\\
\hline
LAAPD glueing (50 pcs): procedure optimization & 5 days  & Genova\\
\hline
LAAPD glueing (200 pcs)& 15 days  & Catania\\
\hline
LAAPD glueing (200 pcs)  & 15 days  & Cagliari\\
\hline
Assembly test: tooling & 10 days  & Genova\\
\hline
Assembly test (50 pcs): procedure optimization & 5 days  & Genova\\
\hline
Assembly test (200 pcs)&  15 days  & Catania\\
\hline
Assembly test (200 pcs)  & 15 days  & Cagliari\\
\hline
Assembly failure&  5 days  & Catania\\
\hline
Assembly failure & 5 days  & Cagliari\\
\hline
\hline
Light Monitoring System & 30 days  & Genova\\
\hline
\hline
\hline
Total&    110 days &Genova \\
\hline
Total&    50 days &Catania \\
\hline
Total&    50 days &Cagliari\\
\hline
Total&    40 days &TorVergata\\
\hline
\hline
\end{tabular}
\end{center}
\label{tb:infn_ts}
\end{table}%
\end{document}


